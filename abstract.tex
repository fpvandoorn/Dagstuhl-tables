\documentclass[a4paper]{article}
\usepackage{hyperref}
\usepackage{amsmath}
\begin{document}

\textbf{The problem statement:} What is the minimum number of meals so
that each of the $n$ conference participants can share at least one
meal with every other participant when eating at tables of at most $k$
persons?  We call this number $T(n,k)$.

In particular, we have an unlimited number of tables, and we do not
require that any two participants have a meal together exactly once,
or that every table is fully occupied.

\medskip

During the seminar, we have made progress on this problem using
various techniques.  This work is being documented via
Github~\cite{dagstuhl-gh}, and is ongoing.

\begin{itemize}
\item We have found several relations between various entries in the
  table of values $T(n,k)$, yielding both lower bounds and upper
  bounds for many entries. These relations allow us to fill in many
  entries in the table without any further exhaustive searches.
\item We have manually computed certain entries $T(n,k)$, allowing us
  to fill in certain regions of the table.
\item We have used Mathematica's built-in SAT solver to compute
  $T(n,k)$ for higher values of $n$ and $k$.
\item We have compared this problem with various related problems,
  such as the Oberwolfach problem~\cite{oberwolfach}, the Social
  Golfer problem~\cite{golf-mathworld,golf-oeis}, and finding Kirkman
  Triple Systems\cite{kirkman-rch}}.  In some cases, this allowed us
to find values $T(n,k)$.
\end{itemize}

As a result of the work, we have submitted sequences to the Online
Encyclopedia of Integer Sequences:
\href{https://oeis.org/draft/A318240}{A318240} and
\href{https://oeis.org/draft/A318241}{A318241}.
We have summarized the results in Table~\ref{tab:dagstuhl}.

\begin{table}
  \centering
\begin{tabular}{@{}c|lllllll@{}}
n / k & 2 & 3 & 4 & 5 & 6 & 7 & 8\tabularnewline
\hline
1 & \textbf{0} & \textbf{0} & \textbf{0} & \textbf{0} & \textbf{0} &
\textbf{0} & \textbf{0}\tabularnewline
2 & \textbf{1} & 1 & 1 & 1 & 1 & 1 & 1\tabularnewline
3 & 3 & \textbf{1} & 1 & 1 & 1 & 1 & 1\tabularnewline
4 & \textbf{3} & 3 cd & \textbf{1} & 1 & 1 & 1 & 1\tabularnewline
5 & 5 & 3 C & 3 d & \textbf{1} & 1 & 1 & 1\tabularnewline
6 & \textbf{5} & 4 e & 3 & 3 d & \textbf{1} & 1 & 1\tabularnewline
7 & 7 & 4 & 3 & 3 & 3 d & \textbf{1} & 1\tabularnewline
8 & \textbf{7} & 4 & 3 B & 3 & 3 & 3 d & \textbf{1}\tabularnewline
9 & 9 & \textbf{4} AH & 4 c & 3 C & 3 & 3 & 3 d\tabularnewline
10 & \textbf{9} & 6 c & 4 E & 4 e & 3 & 3 & 3\tabularnewline
11 & 11 & 6 & 5 e & 4 & 3 & 3 & 3\tabularnewline
12 & \textbf{11} & 6 E & 5 & 4 E & 3 B & 3 & 3\tabularnewline
13 & 13 & 7 a & 5 & 5 e & 4 e & 3 C & 3\tabularnewline
14 & \textbf{13} & 7 & 5 & 5 & 4 & 4 e & 3\tabularnewline
15 & 15 & \textbf{7} F & 5 & 5 & 4 & 4 & 3\tabularnewline
16 & \textbf{15} & 9 c & \textbf{5} H & 5 & 4 & 4 & 3 B\tabularnewline
17 & 17 & 9 & 6-9 a & 5 C & 4 & 4 & 3-4\tabularnewline
18 & \textbf{17} & 9 G & 7-9 a & 5-6 & 4 B & 4 & 3-4\tabularnewline
19 & 19 & 10 a & 7-9 & 5-6 & 5-6 a & 4 C & 3-4\tabularnewline
20 & \textbf{19} & 10 & 7-9 & 5-6 & 5-6 & 4-6 & 4 a B\tabularnewline
21 & 21 & \textbf{10} F & 8-9 c & 6 a & 5-6 & 4-6 & 4-5\tabularnewline
22 & \textbf{21} & 12 c & 8-9 & 6 & 5-6 & 4-6 & 4-5\tabularnewline
23 & 23 & 12 & 8-9 & 6 & 5-6 & 4-6 & 4-5\tabularnewline
24 & \textbf{23} & 12 G & 8-9 & 6 & 5-6 & 5-6 a & 4-5\tabularnewline
25 & 25 & 13 a & 9 a & \textbf{6} AH & 6 c & 5-6 & 4-5\tabularnewline
26 & \textbf{25} & 13 & 9 & 7-9 a & 6 C & 5-6 & 4-5\tabularnewline
27 & 27 & \textbf{13} AH & 9 & 7-9 & 6-7 & 5-6 C & 5 a\tabularnewline
28 & \textbf{27} & 15-16 c & \textbf{9} F & 8-9 a & 6-7 & 5-7 &
5\tabularnewline
29 & 29 & 15-16 & 10-11 a & 8-9 C & 6-7 & 5-7 & 5\tabularnewline
30 & \textbf{29} & 15-16 J & 11 a G & 8-11 & 6-7 B & 5-7 & 5
B\tabularnewline
\end{tabular}
\caption{Table of solutions $T(n,k)$, or ranges of possible
  solutions.  The letters correspond to explanations.  Bold numbers
  are optimal solutions in the sense that every conference
  participants shares a meal with every other participant
  \emph{exactly} once.}
\label{tab:dagstuhl}
\end{table}

\begin{thebibliography}{0}
\bibitem{dagstuhl-gh}
  Floris P. van Doorn, Auke B. Booij.
  \href{https://github.com/fpvandoorn/Dagstuhl-tables/}{\textsl{Dagstuhl's
    Happy Diner problem}}
\bibitem{oberwolfach} Sarah Holliday.  \href{http://facultyweb.kennesaw.edu/shollid4/oberwolfach.php}{\textsl{Sarah's Oberwolfach
    Problem Page}}. Accessed October 2018.
\bibitem{golf-mathworld}
  \href{http://mathworld.wolfram.com/SocialGolferProblem.html}{Social
    Golfer Problem} on Wolfram MathWorld.  Accessed October 2018.
\bibitem{golf-oeis} \href{https://oeis.org/A107431}{A107431} on the
  Online Encyclopedia of Integer Sequences. Accessed October 2018.
\bibitem{kirkman-rch} Dijen K. Ray-Chaudhuri, Richard
  M.\ Wilson. \textsl{Solution of Kirkman's schoolgirl problem}. In
  Proc.\ of Symp.\ in Pure Math, Vol 19, 1971.
\end{thebibliography}
\section{Dagstuhl's Happy Diner
Problem}\label{dagstuhls-happy-diner-problem}

\subsection{The Table Assignment
Assignment}\label{the-table-assignment-assignment}

\textbf{The Problem Statement}: What is the minimum number of meals so
that each of the $n$ conference participants can share at least
one meal with every other participant when eating at tables of at most
$k$ persons? We call this number $T(n,k)$.

In particular, we have an unlimited number of tables, and we do not
require that any two participants have a meal together exactly once, or
that every table is fully occupied.

\subsection{Dagstuhl's Table Table}\label{dagstuhls-table-table}

\begin{tabular}{@{}clllllll@{}}
\hline
n / k & 2 & 3 & 4 & 5 & 6 & 7 & 8\tabularnewline
\hline
1 & \textbf{0} & \textbf{0} & \textbf{0} & \textbf{0} & \textbf{0} &
\textbf{0} & \textbf{0}\tabularnewline
2 & \textbf{1} & 1 & 1 & 1 & 1 & 1 & 1\tabularnewline
3 & 3 & \textbf{1} & 1 & 1 & 1 & 1 & 1\tabularnewline
4 & \textbf{3} & 3 cd & \textbf{1} & 1 & 1 & 1 & 1\tabularnewline
5 & 5 & 3 C & 3 d & \textbf{1} & 1 & 1 & 1\tabularnewline
6 & \textbf{5} & 4 e & 3 & 3 d & \textbf{1} & 1 & 1\tabularnewline
7 & 7 & 4 & 3 & 3 & 3 d & \textbf{1} & 1\tabularnewline
8 & \textbf{7} & 4 & 3 B & 3 & 3 & 3 d & \textbf{1}\tabularnewline
9 & 9 & \textbf{4} AH & 4 c & 3 C & 3 & 3 & 3 d\tabularnewline
10 & \textbf{9} & 6 c & 4 E & 4 e & 3 & 3 & 3\tabularnewline
11 & 11 & 6 & 5 e & 4 & 3 & 3 & 3\tabularnewline
12 & \textbf{11} & 6 E & 5 & 4 E & 3 B & 3 & 3\tabularnewline
13 & 13 & 7 a & 5 & 5 e & 4 e & 3 C & 3\tabularnewline
14 & \textbf{13} & 7 & 5 & 5 & 4 & 4 e & 3\tabularnewline
15 & 15 & \textbf{7} F & 5 & 5 & 4 & 4 & 3\tabularnewline
16 & \textbf{15} & 9 c & \textbf{5} H & 5 & 4 & 4 & 3 B\tabularnewline
17 & 17 & 9 & 6-9 a & 5 C & 4 & 4 & 3-4\tabularnewline
18 & \textbf{17} & 9 G & 7-9 a & 5-6 & 4 B & 4 & 3-4\tabularnewline
19 & 19 & 10 a & 7-9 & 5-6 & 5-6 a & 4 C & 3-4\tabularnewline
20 & \textbf{19} & 10 & 7-9 & 5-6 & 5-6 & 4-6 & 4 a B\tabularnewline
21 & 21 & \textbf{10} F & 8-9 c & 6 a & 5-6 & 4-6 & 4-5\tabularnewline
22 & \textbf{21} & 12 c & 8-9 & 6 & 5-6 & 4-6 & 4-5\tabularnewline
23 & 23 & 12 & 8-9 & 6 & 5-6 & 4-6 & 4-5\tabularnewline
24 & \textbf{23} & 12 G & 8-9 & 6 & 5-6 & 5-6 a & 4-5\tabularnewline
25 & 25 & 13 a & 9 a & \textbf{6} AH & 6 c & 5-6 & 4-5\tabularnewline
26 & \textbf{25} & 13 & 9 & 7-9 a & 6 C & 5-6 & 4-5\tabularnewline
27 & 27 & \textbf{13} AH & 9 & 7-9 & 6-7 & 5-6 C & 5 a\tabularnewline
28 & \textbf{27} & 15-16 c & \textbf{9} F & 8-9 a & 6-7 & 5-7 &
5\tabularnewline
29 & 29 & 15-16 & 10-11 a & 8-9 C & 6-7 & 5-7 & 5\tabularnewline
30 & \textbf{29} & 15-16 J & 11 a G & 8-11 & 6-7 B & 5-7 & 5
B\tabularnewline
\hline
\end{tabular}

Legend: we use lower-case letters $a$-$e$ to justify lower
bounds and upper-case letters $A$-$J$ to justify upper
bounds, which are explained below. No explanation is given when
$n\leq k$ or $k=2$ or the value can be derived from
the inequalities $T(n+1,k)\geq T(n,k)\geq T(n,k+1)$.

The bolded values indicate perfect solutions (see below).

\subsubsection{Dual table}\label{dual-table}

\begin{itemize}
\item
  An entry in this table shows the maximal $n$ such that
  $T(n,k)\leq T$.
\item
  This table has the same information as the previous one, organized
  differently.
\item
  This table is harder to read, but much more informationally dense.
\item
  In this table the upper-case letters show why the entry is not smaller
  (why it is possible to have a solution with this number of
  participants) and the lower-case letters show why the entry is not
  larger (why it is not possible to have a solution with one more
  partipant).
\end{itemize}

\begin{tabular}[]{@{}clllllll@{}}
\hline
T / k & 2 & 3 & 4 & 5 & 6 & 7 & 8\tabularnewline
\hline
1 & \textbf{2} & \textbf{3} & \textbf{4} & \textbf{5} & \textbf{6} &
\textbf{7} & \textbf{8}\tabularnewline
2 & 2 & 3 cd & 4 d & 5 d & 6 d & 7 d & 8 d\tabularnewline
3 & \textbf{4} & 5 C e & 8 B c & 9 C e & 12 B e & 13 C e & 16-19 B
a\tabularnewline
4 & 4 & \textbf{9} AH & 10 E e & 12 E e & 18 B a & 19-23 C a & 20-26 B
a\tabularnewline
5 & \textbf{6} & 9 c & \textbf{16} H a & 17-20 C a & 18-24 c &
&\tabularnewline
6 & 6 & 12 E a & 16-17 a & \textbf{25} AH a & 26-30 C a &
&\tabularnewline
7 & \textbf{8} & \textbf{15} F & 16-20 c & 25-27 a & & &\tabularnewline
8 & 8 & 15 c & 16-24 a & & & \textbf{49} AH a &\tabularnewline
9 & \textbf{10} & 18 G a & \textbf{28} F a & & & &\tabularnewline
10 & 10 & \textbf{21} F & 28-29 a & & & &\tabularnewline
11 & \textbf{12} & 21 c & 32 G & & & &\tabularnewline
12 & 12 & 24 G a & & & & &\tabularnewline
13 & \textbf{14} & \textbf{27} AH & & & & &\tabularnewline
14 & 14 & 27 c & & & & &\tabularnewline
15 & \textbf{16} & 27-30 a & & & & &\tabularnewline
16 & 16 & \textbf{33} F & & & & &\tabularnewline
17 & \textbf{18} & 33 c & & & & &\tabularnewline
18 & 18 & 33-36 a & & & & &\tabularnewline
19 & \textbf{20} & \textbf{39} F & & & & &\tabularnewline
20 & 20 & 39 & & & & &\tabularnewline
21 & \textbf{22} & & \textbf{64} H a & & & &\tabularnewline
22 & 22 & \textbf{45} & & & & &\tabularnewline
23 & \textbf{24} & 45 & & & & &\tabularnewline
24 & 24 & & & & & &\tabularnewline
25 & \textbf{26} & \textbf{51} & & & & &\tabularnewline
26 & 26 & 51 & & & & &\tabularnewline
27 & \textbf{28} & & & & & &\tabularnewline
28 & 28 & \textbf{57} & & & & &\tabularnewline
29 & \textbf{30} & 57 & & & & &\tabularnewline
30 & 30 & & & & & &\tabularnewline
\hline
\end{tabular}

\subsection{Terminology}\label{terminology}

\begin{itemize}
\item
  Given a table assignment for 1 or more meals. We say that this is a
  \emph{valid solution} if every participant meet every other
  participant at least once.
\item
  A valid solution with $n$ participants and table size of at
  most $k$ is called a $(n,k)$-\emph{solution}.
\item
  Given a valid solution. We say that it is an \emph{optimal solution}
  if there is no valid solution (with the same $n$ and
  $k$) with fewer days.
\item
  Given a valid solution. We say that it is a \emph{perfect solution} if
  every participant meets every other participant exactly once.
\item
  The \emph{Social Golfer Problem} is similar: what is the maximum
  possible of meals such that no two participants sit at the same table?
  $G(m,k)$ is the maximal number with $m*k$ participants
  and where each table contains \textbf{exactly} $k$
  participants.
\end{itemize}

\subsection{Properties}\label{properties}

\subsubsection{Perfect Solutions}\label{perfect-solutions}

\begin{itemize}
\item
  Necessarily, every perfect solution is optimal.
\item
  Necessary requirements for a perfect $(n,k)$-solution to exist
  are $k-1|n-1$ and
  $k|n$ (or $n=1$).
\item
  A perfect $(n,k)$-solution exists iff
  $T(n,k)=(n-1)/(k-1)$.
\end{itemize}

\subsubsection{Known Values}\label{known-values}

\begin{itemize}
\item
  $T(n,2)=n$ if $n$ is odd, $T(n,2)=n-1$ if
  $n$ is even.
\item
  The lower bound is given by $c$, see below.
\item
  The upper bound can be obtained as follows.

  \begin{itemize}
  \item
    Suppose $n=4m$: split it up in 2 groups of size
    $2m$, first do those groups independently in $2m-1$
    days, then in the next $2m$ days on day $i$ let person
    $j$ in group 1 sit with person $i+j(\mod2m)$ in
    group 2.
  \item
    Suppose $n=2m$ with $m$ odd:
  \item
    For the first $m$ days: on day $i$ let person
    $i$ sit with $i+m$ and person $i+j$ with
    $i-j$.
  \item
    For the next $m-1$ days, on day $2k-1$ and $2k$
    let person $i$ sit with the persons $i+2k-1$ and
    $i-(2k-1)$. If $i$ is even it will sit with
    $i+2k-1$ first, and if $i$ is odd it will sit with
    $i-(2k-1)$ first.
  \item
    Suppose $n$ is odd, then we can use the solution for
    $n+1$ participants, dropping 1 participant.
  \end{itemize}
\item
  $T(n,k)=1$ if $n\leq k$.
\item
  If $k<n\leq (3/2)k$ then $T(n,k)=3$.
\item
  $d$: It cannot be done in 2 days, because on day 1 there are at
  least 2 tables. All participants on table 1 need to be sit with all
  participants not on table 1 on day 2, but that means that everyone
  needs to sit together on day 2. Contradiction.
\item
  $D$: It can be done in 3 days. Suppose
  $2k-n\geq Ceil(k/2)$, i.e. $2k-n\geq k/2$, i.e.
  $3k\geq 2n$. Then 3 sets of size $k$ can cover all
  pairs.

  \begin{itemize}
  \item
    On day 1 sit participants $1-\/-k$ together.
  \item
    On day 2 sit $(k+1)-\/-n$ and participants
    $1-\/-Ceil(k/2)$ together.
  \item
    On day 3 sit $Ceil(k/2)+1-\/-n$ together.
  \item
    The upper bound $D$ is worse than $B$, which shows
    that for even $k$ we have
    $T(2k,k+1)\leq T(2k,k)\leq T(4,2)=3$.
  \end{itemize}
\item
  If $k$ is prime and $n$ is a power of $k$, then
  there is a perfect $(n,k)$-solution. This follows from upper
  bound $A$ (by induction) or from the next bullet point.
\item
  $H$: If $k$ is a prime power and $n$ is a power
  of $k$, then there is a perfect $(n,k)$-solution.
\item
  Consider the field $F$ of order $k$, and a vector field
  $V$ with cardinality $n$ over $F$.
\item
  For every 1-dimensional subspace $L$ of $V$ the sets of
  1-dimensional affine spaces parallel to $L$ forms a partition
  of $V$. This defines a table assignment for a single meal.
\item
  The set of all table assignments determined by all 1-dimensional
  subspaces in this way forms a perfect $(n,k)$-solution. The
  reason that it is perfect follows from the fact that 1-dimensional
  affine spaces stand in bijective correspondence to pairs of points in
  $V$.
\item
  This idea is due to Neil Strickland.
\item
  $J$: A perfect $(n,3)$-solution for $n\geq 3$ is
  called a \emph{Kirkman Triple System} and is possible iff
  $n\equiv3\mod6$.
\item
  This is (supposed to be) proven in \emph{Solution of Kirkman's
  schoolgirl problem}, Ray-Chaudhuri and Wilson (1971). We couldn't find
  a copy of this paper.
\item
  Together with lower bound $a$, this gives that
  $T(6k+1,3)=T(6k+2,3)=T(6k+3,3)=3k+1$.
\item
  Other known specific values.
\item
  $T(32,4)=11$ is an optimal solution. This follows from
  $G$.
\end{itemize}

\subsubsection{Relations:}\label{relations}

\begin{itemize}
\item
  $T(n+1,k)\geq T(n,k)\geq T(n+1,k+1)\geq T(n,k+1)$.
\item
  If a value in the table can be derived from the first inequality, no
  other explanation is given.
\item
  The second inequality is a special case of upper bound $C$.
\item
  $T(n,k)\leq T(n,m)*T(m,k)$.
\item
  If we have a seating arrangement for $n$ participants at table
  size $m$, then we can give a seating arrangement for table size
  $k$ by simulating tables of size $m$ over
  $T(m,k)$ meals.
\item
  This subsumes the relation $T(n+1,k)\geq T(n,k)\geq T(n,k+1)$
  above since $T(k,k+1)=1$.
\item
  If there is a perfect $(n,m)$-solution and a perfect
  $(m,k)$-solution then there is a perfect
  $(n,k)$-solution.
\end{itemize}

\subsubsection{Lower Bounds:}\label{lower-bounds}

\begin{itemize}

\item
  $T(n,k)\geq (n-1)/(k-1)$ (special case of $a$). Every
  participant can see only $k-1$ participants per meal, and needs
  to see $n-1$ participants.
\item
  $a$: Suppose $n=m*k+l$ with
  $0\leq l<k$. There are $n(n-1)/2$ pairs.
  At most $m*k*(k-1)/2+l*(l-1)/2$ pairs can be formed per meal.
  So $T(n,k)$ is at least equal to the quotient of these 2.
\item
  $c$: If $n=m*k+1$ then this bound is
  $n(n-1)/(m*k*(k-1))=n/(k-1)$. Suppose
  $k-1|n$ (i.e.
  $n\equiv -(k-1)\mod k(k-1)$). Then
  $T(n,k)\geq n/(k-1)+1$, because if it's possible after
  $n/(k-1)$ days, we need to form $m*k*(k-1)/2$ new
  connections every meal. This means that
\item
  Every table needs to be size $k$, except for 1 table of size 1
  every meal.
\item
  Nobody can meet the same person twice. This means that after every
  meal, the number of participants participant A has met is divisible by
  $k-1$, so it can never equal $n-1$.
\item
  $d$: see \emph{Known Values}.
\item
  $e$: proven for this special case, see below. (We don't use
  $e$ if another letter applies.)
\end{itemize}

\subsubsection{Upper Bounds:}\label{upper-bounds}

\begin{itemize}
\item
  $A$: $T(km,k)\leq T(m,k)+m$ if $m$ is coprime
  with $(k-1)!$.
\item
  Divide the participants into $k$ groups of $m$ people.
  On the first $T(m,k)$ days, everyone meets every participant of
  their group.
\item
  Number the participants in each group using the remainder classes
  modulo $m$.
\item
  On the $m$ days after that, on day $i$
  ($0\leq i<m$) make a table with participant $j$
  from the first group, $j+i$ from the second group,
  $j+2i$ from the third group, and so on. If $m$ has no
  divisor smaller than $k$, then every participant will meet
  every participant from another group this way.
\item
  In particular, this shows that if there is a perfect
  $(m,k)$-solution and $m$ is coprime with $(k-1)!$
  then there is a perfect $(km,k)$-solution. In particular, if
  $p$ is prime there is a perfect $(p^k,p)$-solution.
\item
  $B$: $T(nl,kl)\leq T(n,k)$. This can be seen by making
  $n$ groups of $l$ people each and always seating all
  people in a single group together.
\item
  $C$: $T(nl+1,kl+1)\leq T(n,k)$. Same as $B$, but
  make one group size $l+1$.
\item
  $D$: see \emph{Known Values}.
\item
  $E$: found solution for this special case, see below. (We don't
  use $E$ if another letter applies.)
\item
  From a good solution of the social golfer's problem (see External
  Links) we can retrieve a solution to the Happy Diner Problem.
\item
  Denote the solution to the social golfer's problem with $m$
  groups and $k$ golfers per group (so $m*k$ golfers
  total) by $G(m,k)$.
\item
  $F$: If $G(m,k)*(k-1)=m*k-1$ then
  $T(m*k,k)=G(m,k)$, because this gives a perfect
  $(m*k,k)$-solution.
\item
  $G$: If $G(m,k)*(k-1)=m*k-2$ then
  $T(m*k,k)=G(m,k)+1$. This is a lower bound by
  $a$ and a upper bound using the solution to $G(m,k)$:
  take the solution to $G(m,k)$ for the first $G(m,k)$
  meals. Then everyone has seen all other participants, but 1. For the
  last meal, have one table for each of the pair of participants which
  still need to see each other.
\item
  The solutions of the social golfer's problem, can be found at the
  following links:

  \begin{itemize}

  \item
    \href{http://web.archive.org/web/20050308115423/http://www.icparc.ic.ac.uk/~wh/golf/}{Warwick's
    result page (2002)} has various perfect solutions with a small
    number of participants.
  \item
    \href{https://www.metalevel.at/sgp/}{Markus Triska's master thesis
    (2008)} has $G(8,4)=10$.
  \item
    \href{http://www.mathpuzzle.com/MAA/54-Golf\%20Tournaments/mathgames_08_14_07.html}{Edd
    Pegg Jr.'s Math Game page (2007)} has $G(8,3)=11$ and
    $G(7,4)=9$ and $G(9,4)=11$.
  \end{itemize}
\item
  (We don't use $F$ and $G$ if another letter applies.)
\item
  $H$, $J$: see \emph{Known Values}.
\end{itemize}

\subsubsection{Solutions for individual
cases}\label{solutions-for-individual-cases}

\begin{itemize}

\item
  The solution $T(6,3)\geq 4$ is very detailed. Other solutions
  with the same techniques will have much less explanation. So if you
  don't understand the reasoning, read $T(6,3)\geq 4$ first.
\item
  The code using the Mathematica SAT-solver was written by Michael Trott
  and optimized by Floris van Doorn.
\end{itemize}

\paragraph{New terminology}\label{new-terminology}

\begin{itemize}

\item
  A \emph{configuration} is an assignment for the \emph{number} of
  participants to each table (but not stating which participant goes
  where).
\item
  A \emph{distribution} is a seating assignment for a single meal.
\item
  We say that a configuration $C$ is dominated if it has two
  tables with $a$ and $b$ participants and
  $a+b\leq k$.
\item
  In this case, we can merge these two tables and still have a valid
  solution, so we may assume we have a solution without dominated
  configurations.
\end{itemize}

\paragraph{$T(6,3) \geq  4$}

\begin{itemize}

\item
  Suppose there is a solution in 3 days.
\item
  At least 2 days need configuration $(3,3)$.
\item
  The reason is that we need to establish 15 connections between
  participants over 3 days.
\item
  We can establish at most 6 connections during a single day by
  configuration $(3,3)$.
\item
  Configuration $(2,2,2)$ is not dominated, but establishes only
  3 connections
\item
  Any other configuration is dominated by either $(3,3)$ or
  $(2,2,2)$.
\item
  If there is at most 1 day with configuration $(3,3)$, then the
  maximum number of established connections is
  $6+3+3=12<15$, which is not enough.
\item
  Without loss of generality we can assume that the first day has
  configuration $(3,3)$, distributed as $123456$ (i.e.
  $1$, $2$ and $3$ sit together and $4$,
  $5$ and $6$ sit together).
\item
  Now on the other days, we can establish at most 4 connections.
\item
  The reason is that if we use configuration $(3,3)$, then 2
  participants on the first table already sat together on table 1, and
  the same for the second table.
\item
  Therefore, configuration $(3,3)$ gives at most 4 new
  connections.
\item
  We already saw that the only other non-dominated configuration gives
  at most 3 new connections.
\item
  Therefore, the maximal number of connections we can establish is
  $6+4+4=14<15$, which is not enough.
\item
  So there is no valid $(6,3)$-solution with 3 days.
\item
  A similar argument \emph{might} show that $T(12,3)\geq 7$. (but
  $T(18,3)=9$, so it is not generally true that
  $T(6k,3)>3k$.)
\item
  $G(4,3)=4$, i.e.~there is no solution where 12 participants
  sit with different people for 5 days with a table size of 3 which
  might indicate that $T(12,3)\geq 7$.
\end{itemize}

\paragraph{$T(12,3) \leq  6$}

\begin{itemize}
\item
  Solution found by Mathematica SAT-solver:

\begin{verbatim}
159 278 3AC 46B
12C 345 8AB 679
12B 348 79A 56C
168 25A 39B 47C
14A 236 57B 89C
137 249 BC 58 6A
\end{verbatim}
\end{itemize}

\paragraph{$T(10,4) \leq  4$}

\begin{itemize}
\item
  Solution found by hand:

\begin{verbatim}
1234 5678 90
1259 3670 48
1280 4679 35
045 1267 389
\end{verbatim}
\end{itemize}

\paragraph{$T(11,4) \geq  5$}

\begin{itemize}

\item
  Suppose there is a solution in 4 days.
\item
  The only configurations which are not dominated are $(4,4,3)$
  and $(3,3,3,2)$. The first adds at most 15 connections, the
  second at most 10.
\item
  Therefore, on at least 3 days we need a (4,4,3) configuration. WLOG
  day 1 is distributed $12345678ABC$.
\item
  For the other days any table of size 4 has 1 pair in common with day
  1, so adds at most 5 new connections. Therefore, at lost 13 new
  connections can be added during each day.
\item
  This means we cannot get 55 connections, therefore we get a
  contradiction.
\end{itemize}

\paragraph{$T(10,5) \geq  4$}

\begin{itemize}

\item
  Suppose there is a valid solution in 3 days.
\item
  The only configurations which are not dominated are $(5,5)$
  (<= 20 conns), $(4,4,2)$ (<= 13 conns) and
  $(4,3,3)$ (<= 12 conns).
\item
  Therefore, we need $(5,5)$ at least once. WLOG day 1 is
  distributed $0123456789$.
\item
  From now on $(5,5)$ has at most 12 new conns, $(4,4,2)$
  has at most 9 new conns and $(4,3,3)$ has at most 8 new conns.
\item
  This means we cannot get 45 connections, therefore we have no valid
  solution in 3 days.
\end{itemize}

\paragraph{$T(12,5) \leq  4$}

\begin{itemize}
\item
  Solution found by Mathematica SAT-solver:

\begin{verbatim}
12345 6789A BC
1279 38B 456AC
128C 4579B 36A
126AB 379C 458
\end{verbatim}
\end{itemize}

\paragraph{$T(13,5) \geq  5$}

\begin{itemize}

\item
  This solution was found part by hand, part by computer brute-force.
\item
  Suppose there is a valid solution in 4 days.
\item
  $(5,5,3)$ or $(5,4,4)$ has to occur at least once.
\item
  If $(5,5,3)$ never occurs, then from day 2 on at most 18
  connections are possible. Contradiction.
\item
  So day 1 is $(5,5,3)$ (23 connections).
\item
  From then on, at most 19 connections are possible, which has to occur
  at least once, so day 2 is (5,5,3) with 8+8+3 new connections. This
  can be done in 1 way (up to renaming participants)
\item
  Then there are 8 ways for day 3 to have 18 connections (and more is
  impossible), possibly counting things twice. We found this number by
  brute force.
\item
  For none of those 8 ways, there is a valid 4th day.
\end{itemize}

\paragraph{$T(13,6) \geq  4$}

\begin{itemize}

\item
  Suppose there is a valid solution in 3 days.
\item
  The configurations with at least 26 connections which are not
  dominated are $(6,6,1)$ and $(6,5,2)$, one of which has
  to occur at least once.
\item
  Suppose day 1 is $(6,6,1)$. Then no other day can have more
  than 20 connections. Day 2 has

  \begin{itemize}

  \item
    $(6,6,1)$ at most 11+9 = 20 connections
  \item
    $(6,5,2)$ at most 11+8+1 = 20 connections (actually, less)
  \item
    $(6,4,3)$ and $(5,5,3)$ and $(5,4,4)$ also have
    less than 20 connections, all other configurations are dominated.
  \end{itemize}
\item
  Suppose no day is $(6,6,1)$. Then every day needs 26
  connections exactly, which is impossible.
\end{itemize}

\paragraph{$T(14,7)\geq 4$}

\begin{itemize}

\item
  Suppose there is a valid solution in 3 days.
\item
  The configuration $(7,7)$ has to occur, since there is no way
  to make at least 61 connections in 2 days otherwise.
\item
  After $(7,7)$ at most 24 connections can be made per day. So
  there are at most $42+24+24=90<91$
  connections, which is not enough.
\end{itemize}

\subsection{Questions}\label{questions}

\begin{itemize}

\item
  If $n\equiv k\mod k(k-1)$ is there always a perfect
  $(n,k)$-solution? Is it true if we assume $k$ is a prime
  power or a prime number? There is no reason to believe this, but it is
  true for all values where the answer is known.
\item
  It is true for $k=3$. For $k=4$ it's true when
  $n\leq 28$.
\item
  For every $n$ and $k$ is there an optimal
  $(n,k)$-solution in which, during every meal, at most one table
  is not completely occupied?
\item
  This is false. All optimal $(8,5)$-solutions have at least one
  day with two tables of four participants. This was found by brute
  force, but is quite easy to see by hand (to do).
\item
  It is probably even false that there always is an optimal
  $(n,k)$-solution where there are $\lceil n/k\rceil$ tables each
  day (where $\lceil x\rceil$ is the smallest integer which is at least
  $x$). The Mathematica SAT-solver easily found a solution that
  $T(12,3)\leq 6$, but didn't terminate within reasonable time
  when the additional condition was imposed that only 4 tables could be
  used per day.
\end{itemize}

\subsubsection{Conjectures}\label{conjectures}

\begin{itemize}

\item
  $T(n,k)\leq n/(k-1)+O(1)*log(n)$. This should follow
  from an inductive argument using $A$.
\item
  For all $k$, $T(n,k)-n/(k-1)$ is bounded by a
  constant (independent of $n$, possibly dependent on
  $k$).
\item
  This is true for $k=3$. In fact, the optimal
  $(n,3)$-solution is at most 1 higher than the value obtained
  from the lower bound $c$. The reason for this is that for every
  $m$ there is a perfect $(6m+3,3)$-solution (see
  \emph{Known values}), and the lower bound for $6m-2$ given by
  $c$ is only 1 lower than the value for $6m+3$.
\end{itemize}

\subsection{External Links}\label{external-links}

\begin{itemize}

\item
  Dagstuhl's Happy Diner Problem: we are currently writing draft
  sequences \href{https://oeis.org/draft/A318240}{A318240} and
  \href{https://oeis.org/draft/A318241}{A318241}.
\item
  We couldn't find any other place where partial solutions of this
  problem were given.
\item
  On
  \href{http://facultyweb.kennesaw.edu/shollid4/oberwolfach.php}{Sarah's
  Oberwolfach Problem Page} the problem has been stated and finding the
  perfect $(n,3)$-solutions is a special case of the Oberwolfach
  Problem.
\item
  Social Golfer Problem:
\item
  \href{http://mathworld.wolfram.com/SocialGolferProblem.html}{Wolfram
  Mathworld}
\item
  \href{http://web.archive.org/web/20050308115423/http://www.icparc.ic.ac.uk/~wh/golf/}{Warwick's
  result page (2002)}
\item
  \href{https://www.metalevel.at/sgp/}{Markus Triska's master thesis
  (2008)}
\item
  \href{http://www.mathpuzzle.com/MAA/54-Golf\%20Tournaments/mathgames_08_14_07.html}{Edd
  Pegg Jr.'s Math Game page (2007)}
\item
  \href{https://oeis.org/A107431}{A107431}.
\item
  Kirkman Triple System:
\item
  \href{http://mathworld.wolfram.com/KirkmanTripleSystem.html}{Wolfram
  Mathworld},
\item
  \href{https://babel.hathitrust.org/cgi/pt?id=njp.32101065911230;view=1up;seq=19}{Dutch
  dissertation by Pieter Mulder (1917)} (pdf available on request).
\item
  \emph{Solution of Kirkman's schoolgirl problem}, Ray-Chaudhuri and
  Wilson, 1971. In \href{http://www.ams.org/books/pspum/019/}{Proc. of
  Symp. in Pure Math, Vol 19}. (please send pdf if you can access it.)
\item
  \emph{Kirkman triple systems and their generalizations: A survey},
  Rees and Wallis, 2002. (please send pdf if you can access it.)
  \href{https://link.springer.com/chapter/10.1007/978-1-4613-0245-2_13}{Springer}
\item
  Oberwolfach Problem:
  \href{http://facultyweb.kennesaw.edu/shollid4/oberwolfach.php}{Sarah's
  Oberwolfach Problem Page}.
\end{itemize}

\subsection{Contributing}\label{contributing}

\begin{itemize}

\item
  Contributions are welcome! Feel free to add any information. Please
  provide links or justifications of claims you make.
\end{itemize}


\end{document}
